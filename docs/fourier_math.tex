\documentclass[12pt,a4paper]{article}
\usepackage[utf8]{inputenc}
\usepackage[spanish]{babel}
\usepackage{amsmath, amssymb, amsthm}
\usepackage{graphicx}
\usepackage{hyperref}
\usepackage{geometry}
\usepackage{tikz}
\usetikzlibrary{arrows.meta, decorations.pathmorphing, calc, positioning, shapes.geometric}
\usepackage{listings}
\usepackage{xcolor}
\usepackage{tcolorbox}
\tcbuselibrary{listings, skins, breakable}
\usepackage{fancyhdr}
\usepackage{titlesec}
\usepackage{fontawesome5}
\usepackage{mdframed}
\usepackage{float}

\geometry{margin=2.5cm}

%==============================================================================
% ESQUEMA DE COLORES MODERNO
%==============================================================================
\definecolor{primary}{HTML}{0D9488}      % Teal
\definecolor{secondary}{HTML}{F97316}    % Coral/Orange
\definecolor{accent}{HTML}{8B5CF6}       % Purple
\definecolor{dark}{HTML}{1E293B}         % Slate dark
\definecolor{light}{HTML}{F1F5F9}        % Slate light
\definecolor{success}{HTML}{22C55E}      % Green
\definecolor{codegreen}{rgb}{0,0.6,0}
\definecolor{codegray}{rgb}{0.5,0.5,0.5}
\definecolor{codepurple}{rgb}{0.58,0,0.82}
\definecolor{backcolour}{HTML}{F8FAFC}

%==============================================================================
% ESTILO DE CÓDIGO MEJORADO
%==============================================================================
\lstdefinestyle{cppmodern}{
    backgroundcolor=\color{backcolour},
    commentstyle=\color{codegreen}\itshape,
    keywordstyle=\color{primary}\bfseries,
    numberstyle=\tiny\color{codegray},
    stringstyle=\color{secondary},
    basicstyle=\ttfamily\footnotesize,
    breakatwhitespace=false,
    breaklines=true,
    captionpos=b,
    keepspaces=true,
    numbers=left,
    numbersep=8pt,
    showspaces=false,
    showstringspaces=false,
    showtabs=false,
    tabsize=4,
    xleftmargin=15pt,
    framexleftmargin=15pt,
}
\lstset{style=cppmodern}

%==============================================================================
% CAJAS DE CÓDIGO BONITAS
%==============================================================================
\newtcblisting{cppcode}[2][]{
    enhanced,
    breakable,
    colback=backcolour,
    colframe=primary!60!black,
    arc=3mm,
    boxrule=1pt,
    left=5pt,
    right=5pt,
    top=5pt,
    bottom=5pt,
    fonttitle=\bfseries\color{white},
    title={\faCode\hspace{0.5em}#2},
    coltitle=white,
    attach boxed title to top left={yshift=-2mm, xshift=5mm},
    boxed title style={
        colback=primary,
        arc=2mm,
        boxrule=0pt,
    },
    listing only,
    listing options={
        style=cppmodern,
        language=C++,
        #1
    }
}

%==============================================================================
% CAJAS DE CONCEPTO
%==============================================================================
\newtcolorbox{conceptbox}[1]{
    enhanced,
    breakable,
    colback=primary!5,
    colframe=primary,
    arc=3mm,
    boxrule=1.5pt,
    fonttitle=\bfseries\large,
    title={\faLightbulb\hspace{0.5em}#1},
    coltitle=primary,
    attach boxed title to top left={yshift=-3mm, xshift=5mm},
    boxed title style={
        colback=white,
        arc=2mm,
        boxrule=1pt,
        colframe=primary,
    },
}

\newtcolorbox{mathbox}{
    enhanced,
    colback=accent!5,
    colframe=accent!60!black,
    arc=4mm,
    boxrule=1.5pt,
    left=10pt,
    right=10pt,
    top=10pt,
    bottom=10pt,
}

\newtcolorbox{warningbox}[1]{
    enhanced,
    colback=secondary!10,
    colframe=secondary,
    arc=3mm,
    boxrule=1.5pt,
    fonttitle=\bfseries,
    title={\faExclamationTriangle\hspace{0.5em}#1},
}

%==============================================================================
% ESTILOS DE SECCIONES
%==============================================================================
\titleformat{\section}
    {\normalfont\Large\bfseries\color{dark}}
    {\colorbox{primary}{\textcolor{white}{\thesection}}\hspace{0.5em}}
    {0pt}
    {}
    [\vspace{-0.5em}\textcolor{primary}{\rule{\textwidth}{2pt}}]

\titleformat{\subsection}
    {\normalfont\large\bfseries\color{primary}}
    {\thesubsection}
    {1em}
    {}

%==============================================================================
% HYPERREF CONFIG
%==============================================================================
\hypersetup{
    colorlinks=true,
    linkcolor=primary,
    urlcolor=secondary,
    citecolor=accent,
}

%==============================================================================
% HEADER/FOOTER
%==============================================================================
\pagestyle{fancy}
\fancyhf{}
\fancyhead[L]{\textcolor{codegray}{\small Fourier Epicycles}}
\fancyhead[R]{\textcolor{codegray}{\small\thepage}}
\fancyfoot[C]{\textcolor{codegray}{\small Matemáticas + Código}}
\renewcommand{\headrulewidth}{0.5pt}
\renewcommand{\footrulewidth}{0pt}

%==============================================================================
% DOCUMENTO
%==============================================================================
\title{
    \vspace{-1cm}
    {\Huge\bfseries\textcolor{dark}{Epiciclos de Fourier}}
}
\author{\textcolor{dark}{Fourier Epicycles Project} \\ \small\textcolor{codegray}{\faGithub\hspace{0.3em}fourier-epicycles-cpp}}
\date{\textcolor{codegray}{\today}}

\begin{document}

\maketitle
\thispagestyle{empty}

\begin{abstract}
\noindent
\textcolor{primary}{\rule{\textwidth}{1pt}}\\[0.5em]
Este documento explica las \textbf{matemáticas fundamentales} que permiten representar cualquier curva cerrada como una suma de círculos rotantes (\textbf{epiciclos}). Incluye tanto la teoría matemática como el \textbf{código C++} que implementa la animación.\\[0.5em]
\textcolor{primary}{\rule{\textwidth}{1pt}}
\end{abstract}

\tableofcontents
\newpage

%==============================================================================
\section{Introducción: De Ptolomeo a Fourier}

\begin{conceptbox}{La Idea Central}
Cualquier curva cerrada puede descomponerse en una suma de \textbf{círculos rotantes}. Cada círculo gira a una frecuencia diferente, y al sumarlos, se reconstruye la curva original.
\end{conceptbox}

\vspace{0.5cm}

La idea tiene orígenes antiguos: Ptolomeo usó epiciclos para modelar órbitas planetarias. Hoy, con la \textbf{Transformada de Fourier}, podemos aplicar esta técnica a cualquier forma.

\begin{figure}[H]
\centering
\begin{tikzpicture}[scale=1.8]
    % Fondo circular decorativo
    \fill[primary!10] (0,0) circle (2.2);
    
    % Ejes
    \draw[->, thick, dark] (-2.3,0) -- (2.3,0) node[right, dark] {\textbf{Re}};
    \draw[->, thick, dark] (0,-2.3) -- (0,2.3) node[above, dark] {\textbf{Im}};
    
    % Círculo principal (n=1)
    \draw[primary, very thick, dashed] (0,0) circle (1.2);
    \draw[-{Stealth[length=4mm]}, primary, line width=2pt] (0,0) -- (0.85,0.85);
    \node[primary, font=\bfseries] at (0.3,0.6) {$c_1$};
    \node[primary, font=\small] at (-0.9,1.0) {frecuencia 1};
    
    % Círculo secundario (n=2)
    \draw[secondary, thick, dashed] (0.85,0.85) circle (0.6);
    \draw[-{Stealth[length=3mm]}, secondary, line width=1.5pt] (0.85,0.85) -- (1.35,1.15);
    \node[secondary, font=\bfseries] at (1.3,0.8) {$c_2$};
    
    % Círculo terciario (n=3)
    \draw[accent, thick, dashed] (1.35,1.15) circle (0.3);
    \draw[-{Stealth[length=2mm]}, accent, line width=1.2pt] (1.35,1.15) -- (1.55,1.35);
    \node[accent, font=\bfseries] at (1.7,1.1) {$c_3$};
    
    % Punto final
    \fill[dark] (1.55,1.35) circle (3pt);
    \node[dark, font=\bfseries] at (1.9,1.5) {$f(t)$};
    
    % Flecha decorativa
    \draw[->, thick, success!70!black, decorate, decoration={snake, amplitude=2pt}] 
        (1.55,1.35) -- (2.5,1.35) node[right, font=\small] {traza la curva};
\end{tikzpicture}
\caption{\textbf{Epiciclos en acción:} Cada vector rota a su propia frecuencia. La suma de todos da el punto $f(t)$.}
\end{figure}

%==============================================================================
\section{La Matemática: Series de Fourier Complejas}

\subsection{La Fórmula Mágica}

Dada una función periódica $f(t)$ con período $T$:

\begin{mathbox}
\begin{equation}
\boxed{f(t) = \sum_{n=-\infty}^{\infty} c_n \cdot e^{i n \omega t}}
\label{eq:fourier_series}
\end{equation}
\end{mathbox}

\vspace{0.3cm}

\begin{center}
\begin{tikzpicture}
    \node[draw=primary, fill=primary!10, rounded corners, minimum width=3cm, minimum height=1cm] (omega) at (0,0) {$\omega = \frac{2\pi}{T}$};
    \node[right=0.5cm of omega, font=\small] {frecuencia angular};
    
    \node[draw=secondary, fill=secondary!10, rounded corners, minimum width=3cm, minimum height=1cm] (n) at (6,0) {$n \in \mathbb{Z}$};
    \node[right=0.5cm of n, font=\small] {armónico};
\end{tikzpicture}
\end{center}

\subsection{Calculando los Coeficientes}

Los coeficientes $c_n$ se obtienen integrando:

\begin{mathbox}
\begin{equation}
\boxed{c_n = \frac{1}{T} \int_0^T f(t) \cdot e^{-i n \omega t} \, dt}
\label{eq:coefficients}
\end{equation}
\end{mathbox}

\textbf{Interpretación:} Cada $c_n$ es un número complejo que codifica:
\begin{itemize}
    \item[$\bullet$] \textcolor{primary}{\textbf{Amplitud:}} $|c_n|$ = radio del círculo
    \item[$\bullet$] \textcolor{secondary}{\textbf{Fase:}} $\arg(c_n)$ = ángulo inicial del vector
\end{itemize}

%==============================================================================
\section{De la Teoría al Código: DFT}

En la práctica, no tenemos una función continua sino \textbf{puntos discretos} (el contorno de una imagen).

\subsection{Transformada Discreta de Fourier}

Si tenemos $N$ puntos $\{z_0, z_1, \ldots, z_{N-1}\}$ donde $z_k = x_k + iy_k$:

\begin{mathbox}
\begin{equation}
\boxed{C_n = \frac{1}{N} \sum_{k=0}^{N-1} z_k \cdot e^{-i \frac{2\pi n k}{N}}}
\label{eq:dft}
\end{equation}
\end{mathbox}

\subsection{Implementación en C++}

Primero, definimos la estructura para almacenar cada coeficiente:

\begin{cppcode}{Estructura de Datos: FourierCoefficient}
struct FourierCoefficient {
    int frequency;              // n (frecuencia)
    std::complex<double> cn;    // C_n (coeficiente complejo)
    double amplitude;           // |C_n| (radio del circulo)
    double phase;               // arg(C_n) (fase inicial)
    cv::Scalar color;           // Color para visualizacion
};
\end{cppcode}

\vspace{0.3cm}

Usamos \textbf{KissFFT} para calcular la DFT eficientemente en $O(N \log N)$:

\begin{cppcode}{Cálculo de la DFT con KissFFT}
std::vector<FourierCoefficient> computeDFT(
    const std::vector<std::complex<double>>& points) 
{
    size_t N = points.size();
    
    // Crear objeto FFT
    kissfft<double> fft(N, false);  // false = forward transform
    
    // Calcular transformada
    std::vector<std::complex<double>> result(N);
    fft.transform(points.data(), result.data());
    
    // Normalizar y crear coeficientes
    std::vector<FourierCoefficient> coefficients;
    for (size_t n = 0; n < N; ++n) {
        FourierCoefficient coef;
        coef.frequency = (n < N/2) ? n : n - N;  // Frecuencias centradas
        coef.cn = result[n] / static_cast<double>(N);
        coef.amplitude = std::abs(coef.cn);
        coef.phase = std::arg(coef.cn);
        coefficients.push_back(coef);
    }
    
    return coefficients;
}
\end{cppcode}

%==============================================================================
\section{La Animación: Frame por Frame}

\subsection{Algoritmo Principal}

\begin{figure}[H]
\centering
\begin{tikzpicture}[
    node distance=0.8cm,
    block/.style={
        rectangle, draw=primary, fill=primary!10,
        rounded corners, minimum width=4cm, minimum height=1cm,
        font=\small\bfseries, align=center
    },
    arrow/.style={->, >=Stealth, thick, primary}
]
    \node[block] (load) {1. Cargar Imagen};
    \node[block, below=of load] (contour) {2. Extraer Contorno};
    \node[block, below=of contour] (dft) {3. Calcular DFT};
    \node[block, below=of dft] (sort) {4. Ordenar por Amplitud};
    \node[block, below=of sort] (animate) {5. Animar Epiciclos};
    
    \draw[arrow] (load) -- (contour);
    \draw[arrow] (contour) -- (dft);
    \draw[arrow] (dft) -- (sort);
    \draw[arrow] (sort) -- (animate);
    
    % Anotaciones
    \node[right=1cm of contour, font=\footnotesize, text=codegray] {Canny + findContours};
    \node[right=1cm of dft, font=\footnotesize, text=codegray] {KissFFT $O(N \log N)$};
    \node[right=1cm of sort, font=\footnotesize, text=codegray] {Círculos grandes primero};
\end{tikzpicture}
\caption{\textbf{Pipeline de la animación}}
\end{figure}

\subsection{Evaluación en Tiempo $t$}

Para cada frame, calculamos la posición sumando todos los vectores rotantes:

\begin{cppcode}{Evaluación de la Curva en Tiempo t}
std::complex<double> evaluate(
    const std::vector<FourierCoefficient>& coefficients,
    double t)  // t en [0, 2*PI]
{
    std::complex<double> z(0.0, 0.0);
    
    for (const auto& coef : coefficients) {
        // Calcular angulo actual
        double angle = coef.frequency * t + coef.phase;
        
        // Sumar vector rotante
        z += coef.amplitude * std::exp(std::complex<double>(0.0, angle));
    }
    
    return z;
}
\end{cppcode}

\subsection{Dibujando los Epiciclos}

Para visualizar la magia:

\begin{cppcode}{Dibujando los Círculos y Vectores}
void drawEpicycles(cv::Mat& frame, 
                   const std::vector<FourierCoefficient>& coefficients,
                   double t, cv::Point2d center)
{
    cv::Point2d current = center;
    
    for (const auto& coef : coefficients) {
        double angle = coef.frequency * t + coef.phase;
        
        // Calcular siguiente punto
        cv::Point2d next(
            current.x + coef.amplitude * cos(angle),
            current.y + coef.amplitude * sin(angle)
        );
        
        // Dibujar circulo (orbita)
        cv::circle(frame, current, coef.amplitude, 
                   cv::Scalar(100, 100, 100), 1, cv::LINE_AA);
        
        // Dibujar vector (radio)
        cv::line(frame, current, next, coef.color, 2, cv::LINE_AA);
        
        current = next;
    }
    
    // Punto final - aqui se traza la curva!
    cv::circle(frame, current, 3, cv::Scalar(0, 255, 0), -1);
}
\end{cppcode}

%==============================================================================
\section{¿Por Qué Funciona?}

\begin{conceptbox}{Teorema de Fourier}
Cualquier función periódica \textbf{razonablemente suave} puede aproximarse arbitrariamente bien por una suma de exponenciales complejas (senos y cosenos).
\end{conceptbox}

\subsection{Convergencia y Precisión}

\begin{center}
\begin{tikzpicture}
    \node[draw=success, fill=success!10, rounded corners, minimum width=4cm, minimum height=1.2cm, align=center] at (0,0) 
        {\textbf{$N$ coeficientes}\\Reconstrucción perfecta};
    \node[draw=primary, fill=primary!10, rounded corners, minimum width=4cm, minimum height=1.2cm, align=center] at (5,0) 
        {\textbf{$N/2$ coeficientes}\\Buena aproximación};
    \node[draw=secondary, fill=secondary!10, rounded corners, minimum width=4cm, minimum height=1.2cm, align=center] at (10,0) 
        {\textbf{$N/10$ coeficientes}\\Forma reconocible};
\end{tikzpicture}
\end{center}

\vspace{0.5cm}

Al ordenar los coeficientes por amplitud y usar solo los más grandes, podemos crear animaciones que van ``construyendo'' la figura progresivamente.

%==============================================================================
\section{Referencias}

\begin{enumerate}
    \item \textcolor{secondary}{\faYoutube}\hspace{0.3em} \textbf{3Blue1Brown} - ``But what is a Fourier series?'' \\ 
    \url{https://www.youtube.com/watch?v=r6sGWTCMz2k}
    
    \item \textcolor{dark}{\faBook}\hspace{0.3em} Brigham, E.O. - ``The Fast Fourier Transform and Its Applications''
    
    \item \textcolor{dark}{\faGithub}\hspace{0.3em} KissFFT Library - \url{https://github.com/mborgerding/kissfft}
\end{enumerate}

\vspace{1cm}
\begin{center}
\textcolor{primary}{\rule{0.5\textwidth}{2pt}}\\[0.5cm]
\end{center}

\end{document}
