\begin{minipage}[c]{0.35\textwidth}
    Función periódica $f(t)$:
    
    \begin{equation}
    f(t) = \sum_{n=-\infty}^{\infty} c_n \cdot e^{i n \omega t}
    \end{equation}
\end{minipage}
\hfill
\begin{minipage}[c]{0.6\textwidth}
    \begin{tikzpicture}[scale=2.8]
        % --- LEFT: Epicycles (Generator) ---
        % Circle 1
        \draw[primary!80, thin] (0,0) circle (0.8);
        \draw[-{Stealth[length=2mm]}, primary, line width=1.5pt] (0,0) -- (0.56,0.56);
        
        % Circle 2
        \begin{scope}[shift={(0.56,0.56)}]
            \draw[secondary!80, thin] (0,0) circle (0.4);
            \draw[-{Stealth[length=2mm]}, secondary, line width=1.2pt] (0,0) -- (0.35,0.2);
            
            % Circle 3
            \begin{scope}[shift={(0.35,0.2)}]
                \draw[accent!80, thin] (0,0) circle (0.2);
                \draw[-{Stealth[length=1.5mm]}, accent, line width=1pt] (0,0) -- (0.18,0.08);
                
                % Generator Point P (Red)
                \coordinate (P) at (0.18,0.08);
                \fill[secondary] (P) circle (2pt);
            \end{scope}
        \end{scope}
        
        % --- RIGHT: Trace (Result) ---
        \def\xshift{2.2}
        
        % The Curve
        \draw[secondary, thick, smooth] 
            plot[shift={(\xshift,0)}] coordinates {
                (1.09, 0.84) (1.2, 0.5) (0.8, 0.0) 
                (0.4, 0.2) (0.1, 0.6) (0.3, 1.0) 
                (0.8, 1.1) (1.09, 0.84)
            };
            
        % Result Point P'
        \coordinate (P_prime) at (1.09+\xshift, 0.84);
        \fill[secondary] (P_prime) circle (2pt);
        \node[secondary, font=\bfseries, anchor=west] at (1.15+\xshift, 0.84) {$f(t)$};
        
        % --- CONNECTION (Mirror Line) ---
        \coordinate (P_abs) at (1.09, 0.84); 
        \draw[secondary, dashed, thick] (P_abs) -- (P_prime);
        
    \end{tikzpicture}
\end{minipage}

\vspace{0.3cm}

\textbf{Estructura de Datos ($c_n$)}
\begin{lstlisting}[aboveskip=2pt]
struct FourierCoefficient {
    int freq; double amp, phase; 
    complex<double> cn;
};
\end{lstlisting}

\textbf{Función $f(t)$ (Suma)}
\begin{lstlisting}[aboveskip=2pt]
complex<double> eval(const vector<FourierCoefficient>& F, double t) {
    complex<double> z(0,0);
    for (const auto& c : F) // Suma vectorial
        z += c.amplitude * polar(1.0, c.freq*t + c.phase);
    return z;
}
\end{lstlisting}

\vspace{0.8cm}

\subsection{Cálculo de Coeficientes (DFT)}

\begin{equation}
C_n = \frac{1}{N} \sum_{k=0}^{N-1} z_k \cdot e^{-i \frac{2\pi n k}{N}}
\end{equation}

Cada $c_n$ se obtiene "promediando" la rotación inversa de los puntos.

\begin{lstlisting}[aboveskip=2pt]
vector<FourierCoefficient> computeDFT(const vector<complex<double>>& pts) {
    size_t N = pts.size();
    vector<complex<double>> res(N);
    kissfft<double> fft(N, false);
    fft.transform(pts.data(), res.data());
    // ... mapeo a struct ...
    return coeffs;
}
\end{lstlisting}
